\RequirePackage{fix-cm}
% oneside : 단면 인쇄용
% twoside : 양면 인쇄용
% ko : 국문 논문 작성
% master : 석사
% phd : 박사
% openright : 챕터가 홀수쪽에서 시작
\documentclass[oneside,ko,master]{snuthesis_utf8}

%%%%%%%%%%%%%%%%%%%%%%%%%%%%%%%%%%%%%%%%
%% 목차 양식을 변경하는 코드
%% subfigure (subfig) package 사용 여부에 따라
%% tocloft의 옵션을 다르게 지정해야 한다.
%\usepackage[titles,subfigure]{tocloft} % when you use subfigure package
\usepackage[titles]{tocloft} % when you don't use subfigure package
\makeatletter % don't delete me
\if@snu@ko
	\renewcommand\cftchappresnum{제~}
	\renewcommand\cftchapaftersnum{~장}
	\renewcommand\cftfigpresnum{그림~}
	\renewcommand\cfttabpresnum{표~}
\else
	\renewcommand\cftchappresnum{Chapter~}
	\renewcommand\cftfigpresnum{Figure~}
	\renewcommand\cfttabpresnum{Table~}
\fi
\makeatother % don't delete me
\newlength{\mytmplen}
\settowidth{\mytmplen}{\bfseries\cftchappresnum\cftchapaftersnum}
\addtolength{\cftchapnumwidth}{\mytmplen}
\settowidth{\mytmplen}{\bfseries\cftfigpresnum\cftfigaftersnum}
\addtolength{\cftfignumwidth}{\mytmplen}
\settowidth{\mytmplen}{\bfseries\cfttabpresnum\cfttabaftersnum}
\addtolength{\cfttabnumwidth}{\mytmplen}
%% 목차 양식을 변경하는 코드 끝
%%%%%%%%%%%%%%%%%%%%%%%%%%%%%%%%%%%%%%%%

%%%%%%%%%%%%%%%%%%%%%%%%%%%%%%%%%%%%%%%%
%% 다른 패키지 로드
%% http://faq.ktug.or.kr/faq/pdflatex%B0%FAlatex%B5%BF%BD%C3%BB%E7%BF%EB
%% 필요에 따라 직접 수정 필요
\ifpdf
	%\input glyphtounicode\pdfgentounicode=1 %type 1 font사용시
	%\usepackage[pdftex,unicode]{hyperref} % delete me
	%\usepackage[pdftex]{graphicx}
	%\usepackage[pdftex,svgnames]{xcolor}
\else
	%\usepackage[dvipdfmx,unicode]{hyperref} % delete me
	%\usepackage[dvipdfmx]{graphicx}
	%\usepackage[dvipdfmx,svgnames]{xcolor}
\fi
%%%%%%%%%%%%%%%%%%%%%%%%%%%%%%%%%%%%%%%%

%% \title : 22pt로 나오는 큰 제목
%% \title* : 16pt로 나오는 작은 제목
\title{논문 제목}
\title*{TITLE OF THE THESIS}

%% 저자 이름 Author's(Your) name
\author{홍길동}
\author*{홍~길~동} % Insert space for Hangul name.
%\author{Gildong Hong}
%\author*{Gildong Hong} % Same as \author.

%% 학번 Student number
\studentnumber{2000-00000}

%% 지도교수님 성함 Advisor's name
\advisor{홍길동}
\advisor*{홍~길~동} % Insert space for Hangul name.
%\advisor{Gildong Hong}
%\advisor*{Gildong Hong}

%% 학위 수여일 Graduation date
%% 표지에 적히는 날짜.
%% 학위 수여일이 아니라 논문 발간년도를 적어야 할 수도 있음.
\graddate{2013~년~2~월}
%\graddate{FEBRUARY 2010}

%% 논문 제출일 Submission date
\submissiondate{2012~년~11~월}

%% 논문 인준일 Approval date
\approvaldate{2012~년~12~월}

%% 참고: 인준지의 교수님 성함은
%% 컴퓨터로 출력하지 않고, 교수님께서
%% 자필로 쓰시기도 합니다.
%% 위원장, 부위원장, 위원
\committeemembers%
{교수님 1}%
{교수님 2}%
{교수님 3}%
{교수님 4}%
{교수님 5}
%% 밑줄 길이
%\setlength{\committeenameunderlinelength}{7cm}

\begin{document}
\pagenumbering{Roman}
\makefrontcover
\makefrontcover
\makeapproval

\cleardoublepage
\pagenumbering{roman}

\keyword{서울대학교, 컴퓨터 공학부, 졸업논문}
\begin{abstract}
요약 내용이 여기 들어갑니다.
\end{abstract}

\cleardoublepage
\addcontentsline{toc}{chapter}{\contentsname}
\tableofcontents

\cleardoublepage
\addcontentsline{toc}{chapter}{\listfigurename}
\listoffigures

\cleardoublepage
\addcontentsline{toc}{chapter}{\listtablename}
\listoftables

\cleardoublepage
\pagenumbering{arabic}

\chapter{서론}
서론.

\section{section 제목}
1.1 내용이 여기에 들어갑니다.

\subsection{하위 section 제목}
1.1.1 내용이 여기에 들어갑니다.

\begin{table}[!hbp]
\caption{표 제목이 여기에 들어갑니다.}
\end{table}

\subsection{하위 section 제목}
1.1.2 내용이 여기에 들어갑니다.

\begin{figure}[!hbp]
\caption{그림 제목이 여기에 들어갑니다.}
\end{figure}

\chapter{본론}
2장 본론 내용이 여기에 들어갑니다.

\chapter{결론}
3장 결론 내용이 여기에 들어갑니다.

\begin{thebibliography}{00}
\addcontentsline{toc}{chapter}{\bibname}

% 영문저널의 경우
    \bibitem{ref1} B. Jeon and J. Jeong, ``Blocking artifacts
    reduction in image compression with block boundary discontiunity
    criterion,'' {\em IEEE Transactions on Circuits and Systems for
    Video Tech.}, vol. 8, no.3, pp. 345-357, June 1998.

% 영문학술대회의 경우
    \bibitem{ref2} W. G. Jeon and Y. S. Cho, ``An equalization
    technique for OFDM and MC-CDMA in a multipath fading channels,''
    in {\em Proceedings of IEEE Conference on Acoustics, Speech and
    Signal Processing}, Munich, Germany, May 1997. pp. 2529-2532.

% 국내학술대회의 경우
    \bibitem{ref3} 윤남국, 김수종, ``무선 센서 네트워크에서의 에너지
    효율적인 그라디언트 기반 라우팅 기법,'' {\em 한국정보과학회
    2006년 추계학술대회}, 제12권, 제2호, 2006년 10월. pp.
    1372-1374.

% 단행본의 경우
    \bibitem{ref4} C. Mead and L. Conway, {\em Introduction to VLSI
    Systems}, Addison-Wesley, Boston, 1994.

% URL
    \bibitem{ref5} The SolarMESH Network,
    http://owl.mcmater.ca/solarmesh

% Technical Report의 경우
    \bibitem{ref6} K. E. Elliott and C. M. Greene, ``A local adaptive
    protocol,'' Argonne National Laboratory, Argonne, France,
    Technical Report 916-1010-BB, 1997.

% 학위논문의 경우
    \bibitem{ref7} T. Kim, ``Scheduling and Allocation Problems in
    High-level Synthesis,'' Ph. D. Dissertation, ECE Department,
    Univ. of Illinois at U-C, 1993.

% 특허의 경우
    \bibitem{ref8} Sunghyun Choi, ``Wireless MAC protocol based on a
    hybrid combination of slot allocation, token passing, and
    polling for isochronous traffic,'' U.S. Patent No. 6,795,418,
    September 21, 2004.

% 표준
    \bibitem{ref9} IEEE Std. 802.11-1999, Part 11: Wireless LAN
    Medium Access Control (MAC) and Physical Layer (PHY)
    specifications, Reference number ISO/IEC 8802-11:1999(E), IEEE
    Std. 802.11, 1999 edition, 1999.

\end{thebibliography}

\keywordalt{SNU, Computer Science Engineering, thesis}
\begin{abstractalt}
\abstracttitle{Title of The Thesis}
\abstractauthor{Gildong Hong}
Abstract
\end{abstractalt}

\acknowledgement
감사의 글이 여기 들어갑니다.

\end{document}

